\chapter{Introduction}
\label{cha:intro}


% SECTION 1
\section{Microbial communities: structure \& function}

Microbes, i.e. Bacteria, Archaea and small Eukaryotes such as protoza, are omnipresent and impact global ecosystem functions \citep{falkowski2008microbial} through their abundance \citep{bar2018biomass}, versatility \citep{rees2017improving} and interactions \citep{rottjers2018hairballs}. 


% These facts have inspired microbiologists from diverse scientific fields to study their genotype and phenotype [5], their metabolism [6] and their interactions with the environment [7].



\subsection{The role of microbial communities in biogeochemical cycles}


\subsection{Microbial interactions: unravelling the microbiome}

% \subsection{Methods for microbial interactions inference}

% \subsection{Competition and mutualism: a dialectic relationship}


% SECTION 2
\section{Bioinformatics challenges in HTS approaches}



% SECTION 3
\section{Data integration \& data mining in the era of omics}


% SECTION 4
\section{Sampling the flux space of a metabolic model: challenges \& potential}



% SECTION 5
\section{The hypersaline Tristomo swamp: a case study of an extreme environment}


% SECTION 6
\section{Systems biology from a computational resources point-of-view}





% SECTION 7
\section{Aims and objectives}

The aim of this PhD was double; 
on the one hand, to enhance the analysis of microbiome data by building algorithms and software to address some of the on-going computational challenges on the field.
On the other, to exploit these methods to identify taxa, functions, especially related to sulfur cycle, and microbial interactions that support life in microbial community assemblages in hypersaline sediments.
All parts of this work are computational. 

In \textbf{Chapter 2}, challenges derived from the analysis of HTS amplicon data are examined.
A bioinformatics pipeline, called \texttt{pema}, for the analysis of several marker genes was developed, combinining several new technologies that allow large scale analysis of hundreds of samples. 
In addition, a software tool called \texttt{darn}, was built to investigate the unassigned sequences in amplicon data of the COI marker gene. 

In \textbf{Chapter 3}, data integration, data mining and text-mining methods were exploited to build a knowledge-base, called \texttt{prego}, including millions of associations between:
\begin{enumerate}
   \item microbial taxa and the environments they have been found in 
   \item microbial taxa and biological processes they occur
   \item environmental types and the biological processes that take place there
\end{enumerate}

In \textbf{Chapter 4}, the challenges of flux sampling in metabolic models of high dimensions was presented along with a Multiphase Monte Carlo Sampling (MMCS) algorithm we developed. 

In \textbf{Chapter 5}, sediment samples from a hypersaline swamp in Tristomo, Karpathos Greece were analysed using both amplicon and shotgun metagenomics. 
The taxonomic and the functional profiles of the microbial communities present there were investigated. 
Key microbial interactions for the assemblages were infered. 
All the methods developed and presented in the previous chapters were used to enhance the analysis of this microbiome.

In \textbf{Chapter 6}, the history of the IMBBC-HCMR HPC facility was presented indicating the vast needs of computing resources in modern analyses in general and in microbial studies more specifically. 


Finally, in the \textbf{Conclusions} chapter, general discussion and conclusions that have derived from this research were presented. 

