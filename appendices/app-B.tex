\chapter{Computational Geometry}
\label{app:comp_geom_intro}

\section{Definitions \& concepts}



   A \textit{hyperplane} is a set of the form
   \begin{equation}
      H = \{x \in \mathbb{R}^n : p^{T} \cdot x = t \}      
   \end{equation}
   and defines two closed \textit{halfspaces}.
   Such a halfspaces would be denoted as
   \begin{equation}
      \begin{split}
         H_{-} = \{ x \in \mathbb{R}^n : p^{T} \cdot x \leq t \} \\
         H_{+} = \{ x \in \mathbb{R}^n : p^{T} \cdot x \geq t \}               
      \end{split}
   \end{equation}

   The intersection of a finite number of halfspaces builds a \textit{polyhedron}.
   A system of inequalities arises:
   \begin{equation}
      a_{i}^{t} \cdot x \leq b_{i},  i \in \{1, .., m\}      
   \end{equation} 
   where $m$ is the number of halfspaces. 
   Thus, a polyhedron can be denoted as:
   \begin{equation}
      P = \{ x \in \mathbb{R}^n : A \cdot x \leq b \}
   \end{equation}
   where $A$ is a $m*n$ matrix
   with $m$ being the number of halfspaces 
   and $n$ the dimension of the space. 
   Finally, $b$ is a vector of the right side of the inequalities ($b_{i}$).
   
   A \textit{bounded} polyhedron meaning, 
   $ \exists M > 0$ such that $\|x\| \leq M$ for all $x \in P$,
   is called a \textit{\textbf{polytope}}.
   
   Some of the inequalities in $A$ though can be geometrically redundant,
   meaning that if these are removed $P$ remains the same. 
   The \textit{dimension} of a polytope $P$ is equal to $n - r(P)$
   where $r(P)$ is the maximum number of linearly independent 
   defining hyperplanes containing $P$.

   We call \textit{defining hyperplanes} the \textbf{total}
   hyperplanes defined from the system, meaning 
   those coming from the $A \cdot x \leq 0$ 
   plus those coming from any constraints. 
   For example, maybe we would have $x \geq 0$ then these hyperplanes would be also considered
   as defining hyperplanes.

   We consider $P$ as a \textit{\textbf{fully dimensional}} polytope
   if and only if $dim(P) = n$.
   In other words, 
   a $d$-polytope is full-dimensional in $d$-space.
   Each (nonredundant) inequality corresponds to a facet of the polytope

   In case that our system has only inequalities, 
   then the polytope derived is always full dimensional.
   However, in case that extra cosntraints as equalities
   are included, then the polytope derived could be 
   full-dimensional or not. 
   If the space defined by the equalities intersects the one 
   defined by the inequalities, then the polytope is not full-dimensional.

   
   A \textit{face} is a set of points $ F \subseteq P$ that belongs 
   to the intersection of a nonempty set of defining hyperplanes 
   To show that a valid inequality is a face we just need to 
   find a point in the intersection of the hyperplane it defines
   and our polytope. 
   To show that a face is a \textit{\textbf{facet}}, 
   i.e. a face of dimension $n -1$,
   we need to show that it belongs to exactly one defining hyperplane.
   If it belongs to more, then it is no longer a facet.

   \paragraph{Facets are necessary and sufficient} for the complete description of a polytope 
   in terms of valid inequalities.

   % ----------------------------------------

   % \rule{12cm}{0.25mm} 


      If $P$ is full-dimensional then it has a unique minimal description: 
      \begin{equation}
         P = \{ x \in \mathbb{R}^n : a_{i}^T \cdot x \leq b_{i}, i = \{1, ..m\} \}
      \end{equation}
      where each of the $m$ inequalities is unique to within a positive multiple.
   

   % \rule{12cm}{0.25mm} 

   % -----------------------------------------

   Points $ x_{1}, x_{2}, ... , x_{k} \in \mathbb{R}^n$ are \textit{affinely independent}
   if the $k-1$ directions: 
   $x_{i} - x_{1}, i \in \{2, k\}$
   are linearly independent. 
   The maximum number of affinely independent points in $P$ is denoted as $i(P)$. 
   Now the dimension of $P$ can be defined as:
   $dim(P) = i(P) - 1$

   To show that $P$ is full-dimensional we just need to 
   show that it has exactly $n+1$ affinely independent points. 

   A matrix is said to have full rank if its rank equals the largest possible for a matrix of the same dimensions, which is the lesser of the number of rows and columns. 
   % We compute the rank using echelon forms, 
   % singular value decomposition (SVD), 
   % QR decomposition with pivoting 




   % every time we add a hyperplane, we loose one degree of freedom 


\section*{Markov Chain Monte Carlo}
\label{  }

      \begin{definition}
         A Markov chain or Markov process is a stochastic model describing a sequence of possible events in which the probability of each event depends only on the state attained in the previous event
      \end{definition}
   
      Markov Chain Monte Carlo (MCMC) methods are algorithms that sample from a probabilistic distribution. 


